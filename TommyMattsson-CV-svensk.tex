% LaTeX file for resume
% This file uses the resume document class (res.cls)

\documentclass[margin]{res}
\usepackage[utf8]{inputenc}
% the margin option causes section titles to appear to the left of body text
\textwidth=5.2in % increase textwidth to get smaller right margin
%\usepackage{helvetica} % uses helvetica postscript font (download helvetica.sty)
%\usepackage{newcent}   % uses new century schoolbook postscript font

\begin{document}

\name{CV - Tommy Mattsson\\[12pt]} % the \\[12pt] adds a blank line after name

\address{{\bf Kontaktinformation}\\
  \input{contact.tex}
  linkedin.com/in/tommymattsson \\
  github.com/Taddic
}

\address{{\bf Adress}\\
  \input{address.tex}\\
}

\begin{resume}

\section{Jobberfarenhet}
{\bf Mjukvaruutvecklare,} Telia Company, Uppsala, Sverige \hfill 2017-02 -- Pågående
 \begin{itemize} \itemsep -2pt  % reduce space between items
 \item Utveckla mjukvara primärt i Erlang och Riak.
 \item Utveckla en process för automatiska tester skrivna i Erlang som körs i Jenkins.
 \item Bidra till utvecklingen mot Docker/containers.
 \item Sköta om ett par Jenkins instanser.
 \end{itemize}

{\bf Support Engineer,} Cisco Systems, Stockholm, Sverige \hfill 2014-08 -- 2017-01
 \begin{itemize} \itemsep -2pt  % reduce space between items
 \item Samma position som hos Tail-f Systems, företaget blev uppköpt.
 \end{itemize}

{\bf Support Engineer,} Tail-f Systems, Stockholm, Sverige \hfill 2014-03 -- 2014-07
 \begin{itemize} \itemsep -2pt  % reduce space between items
 \item Bistå med teknisk support fór mjukvarorna ``ConfD'' och ``NCS''
   utvecklade av Tail-f systems.
 \item Arbeta med språk och verktyg såsom YANG modelering, NETCONF,
   XML, XPath, Erlang, C, Java, Python.
 \end{itemize}

{\bf Master examensarbete,} Ericsson, Shanghai, Kina \hfill 2013-01 -- 2013-07
 \begin{itemize} \itemsep -2pt  % reduce space between items
 \item Titel för examensarbetet var ``Content filtering''. Uppdraget var
   att designa och implementera ICAP (IETF RFC 3507) i Erlang för
   använding i ett verktyg för lasttester utvecklat av Ericsson.
 \item Studerade telekommunikationsarkitektur, exempelvis: \\
 GPRS, WCDMA, LTE, SGSN, GGSN, 3G, 2G, PGW, SGW, MME.
 \end{itemize}

{\bf IT-Support,} Uppsala Universitet, Uppsala, Sverige \hfill 2010-10 -- 2012-08
 \begin{itemize} \itemsep -2pt  % reduce space between items
 \item Hjälpa studenter med IT relaterade problem rörande UNIX och
   Windows systemen i datasalarna vid IT institutionen på Uppsala Universitet.
 \item Förbereda ny hårdvara. Installera operativsystem och mjukvara i UNIX/Linux, Mac och Windwos system.
 \end{itemize}

{\bf Mjukvaruutvecklare,} Klarna AB, Stockholm, Sverige \hfill 2009-06 -- 2009-10\\
{\bf Mjukvaruutvecklare,} Klarna AB, Stockholm, Sverige \hfill 2008-06 -- 2008-08
\begin{itemize} \itemsep -2pt  % reduce space between items
\item Utveckla funktionalitet i Erlang till Klarnas' online fakturasystem.
\item Sommarjobb under studietiden.
\end{itemize}

\section{Utbildning}
{\bf M.Sc,  Datavetenskap,} Uppsala Universitet, Uppsala, Sverige \hfill January 2014 \\
Samarbete med Tongji Universitet, Shanghai, Kina. \\
Fokus på nätverk, distribuerade system och mjukvaruutveckling.

{\bf M.Sc, Mjukvaruutveckling,} Tongji Universitet, Shanghai, Kina \hfill June 2013 \\
Samarbete med Uppsala Universitet, Uppsala, Sverige. \\
Inriktning mjukvaruutveckling och Mandarin.

{\bf B.Sc, Datavetenskap,} Uppsala Universitet, Uppsala, Sverige \hfill Maj 2012 \\
Generell datavetenskap.

\newpage
\section{Idell \\ verksamhet}

{\bf Styrelsemedlem,} Uppsala Gymnastikförening, Uppsala, Sverige \hfill 2014 -- 2019
\begin{itemize} \itemsep -2pt
\item IT ansvarig i styrelsen.
\item Vara med på styrelsemöten och utföra tilldelade uppgifter.
\end{itemize}

{\bf Gymnastiktränare,} Uppsala Gymnastikförening, Uppsala, Sverige \hfill 2014 -- 2019
\begin{itemize} \itemsep -2pt
\item Planera och hålla i gymnastikpass för vuxna i allmän gymnastik.
\item Planera och hålla i gymnastikpass för barn i manlig artistisk gymnastik (MAG).
\end{itemize}

\section{Språkkunskaper}
\begin{itemize} \itemsep -2pt
\item Svenska (modersmål)
\item Engelska (flytande i tal och skrift)
\item Franska (grundkunskaper)
\end{itemize}

% Tabulate Computer Skills; p{3in} defines paragraph 3 inches wide
\section{Tekniska kunskaper}
\begin{itemize} \itemsep -2pt
\item Erlang, Python, JavaScript/NodeJS, C/C++, HTML
\item Bash script, Makefile, Docker
\item LaTeX
\item Emacs, IntelliJ
\item Git, SVN, Jenkins
\end{itemize}

\section{Personliga intressen}
Privat tränar jag parakrobatik/acroyoga, gymnastik/calisthenics.
Spelar tv/datorspel, snickrar när tillfälle ges och har även certifikat för att hoppa fallskärm.

\section{Referenser}
Tillhandahålls vid förfrågan.

\end{resume}
\end{document}
