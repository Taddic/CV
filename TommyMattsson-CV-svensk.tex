% LaTeX file for resume
% This file uses the resume document class (res.cls)

\documentclass[margin]{res}
\usepackage[utf8]{inputenc}
% the margin option causes section titles to appear to the left of body text
\textwidth=5.2in % increase textwidth to get smaller right margin
%\usepackage{helvetica} % uses helvetica postscript font (download helvetica.sty)
%\usepackage{newcent}   % uses new century schoolbook postscript font

\begin{document}

\name{Tommy Mattsson\\[12pt]} % the \\[12pt] adds a blank line after name

\address{{\bf Kontaktinformation}\\
\input{contact.tex}
linkedin.com/in/tommymattsson\\
github.com/Taddic\\
}

\address{{\bf Adress}\\
  \input{address.tex}\\
}

\begin{resume}

%\section{Objective}
%Auditing/Analysis of Operations

\section{Utbildning}
{\bf M.Sc,  Datavetenskap,} Uppsala Universitet, Uppsala, Sverige \hfill \textbf{Pågående} \\
Sino-Swedish Datavetenskap och mjukvaruutveckling, samarbete med Tongji Universitet, Shanghai, Kina. Inkluderar ett års studier i Shanghai. \\
Fokus på nätverk, distribuerade system och mjukvaruutveckling.

{\bf M.Sc, Mjukvaruutveckling,} Tongji Universitet, Shanghai, Kina \hfill \textbf{Pågående} \\
Sino-Swedish Datavetenskap och mjukvaruutveckling, samarbete med Uppsala Universitet, Uppsala, Sverige. \\
Fokus på mjukvaruutveckling och kinesiskt språk.

{\bf B.Sc, Datavetenskap,} Uppsala Universitet, Uppsala, Sverige \hfill Maj 2012 \\
Generell datavetenskap.

\section{Jobberfarenhet}
{\bf Master examensarbete,} Ericsson, Shanghai, Kina \hfill 2013-01 -- 2013-07
 \begin{itemize} \itemsep -2pt  % reduce space between items
 \item Titel för examensarbetet var "Content filtering". Uppdraget var att designa och implementera Internet Content Adaptation Protocol (ICAP, IETF RFC 3507) för använding i ett verktyg för belastningstester utvecklat av Ericsson.
 \item Implentationen gjordes i Erlang.
 \item Studerade telekommunikationsarkitektur, exempelvis: \\
 GPRS, WCDMA, LTE, SGSN, GGSN, 3G, 2G, PGW, SGW, MME
 \end{itemize}

{\bf Lärarassistent,} Uppsala Universitet, Uppsala, Sverige \hfill 2013-06 -- 2013-08\\
{\bf Lärarassistent,} Uppsala Universitet, Uppsala, Sverige \hfill 2012-06 -- 2012-08
 \begin{itemize} \itemsep -2pt  % reduce space between items
 \item Lärarassistent till en kurs vid namn "Peer-To-Peer computing".
 \item Förbättra tidigare laborationer skrivna i Erlang med en bättre struktur och användargränssnitt.
 \item Introducera Erlang för studenterna.
 \item Närvara vid laborationstillfällen och svara på e-post med frågor från studenterna.
 \item Rätta laborationsuppgifter.
 \end{itemize}

{\bf IT-Support,} Uppsala Universitet, Uppsala, Sverige \hfill 2010-10 -- 2012-08
 \begin{itemize} \itemsep -2pt  % reduce space between items
 \item Hjälpa studenter med IT relaterade problem, i huvudsak rörande UNIX och Windows systemen i datasalarna på IT institutionen av Uppsala Universitet.
 \item Underhålla skrivare hos IT institutionen.
 \item Förbereda ny hårdvara. Installera operativsystem och mjukvara i UNIX/Linux, Mac och Windwos system.
 \end{itemize}

{\bf Systemutvecklare,} Klarna AB, Stockholm, Sverige \hfill 2009-06 -- 2009-10\\
{\bf Systemutvecklare,} Klarna AB, Stockholm, Sverige \hfill 2008-06 -- 2008-08
 \begin{itemize} \itemsep -2pt  % reduce space between items
 \item Utveckla funktionalitet i Klarnas' online fakturasystem.
 \item Utvecklandet gjordes i programmeringsspråket Erlang.
 \item Andra sommaren fick jag utveckla lite mer avancerade funktioner och hade lite mer ansvar. De kände att mina programmeringsfärdigheter hade blivit bättre och att jag kunde klara mig lite mer på egen hand.
 \end{itemize}

\newpage

\section{Projekt}
{\bf Sensorsökmotor,} Uppsala Universitet, Uppsala, Sverige \hfill 2013-09 -- 2014-01
 \begin{itemize} \itemsep -2pt  % reduce space between items
 \item Utfördes i kursen "Projekt Datavetenskap" till min masterexamen. Målet med kursen var att vi skulle få känna på hur det är att jobba i industrin. Kunden var Ericsson och vi hade även en expert till hjälp från Sics.
 \item Implementation gjordes i huvudsak i Erlang och Ruby on Rails.
 \item Open source projekt på github. \\
 https://github.com/projectcs13/sensor-cloud\\
 https://github.com/projectcs13/sensor-cloud-website
 \end{itemize}


\section{Språkkunskaper}
   \begin{tabular}{l p{3in}}
    Svenska:  & Modersmål \\
    Engelska: & Avancerad förståelse  \\
    Franska:  & Grundläggande förståelse \\
    Kinesiska:& Grundläggande förståelse
 \end{tabular}

% Tabulate Computer Skills; p{3in} defines paragraph 3 inches wide
\section{Programmerings- \\ kunskaper}
   \begin{tabular}{l p{3in}}
    Funktionell programmering: & Erlang, Haskell, Lisp \\
    Imperativ programmering: & Java, Python, PHP, C, C++ \\
    Web programmering: & JavaScript, JQuery, HTML, CSS, Ruby on Rails \\
    UNIX/Linux: & Bash script, Makefile \\
    App programmering: & Android API \\
    Annat: & LaTeX (denna CV är gjord med hjälp av LaTeX)
 \end{tabular}

\section{Mjukvara}
\begin{tabular}{l p{3in}}
	IDE: & Eclipse, Netbeans \\
	CVS: & Git, SVN \\
	Webservrar: & Apache, YAWS, Mochiweb \\
	Editorer: & Emacs \\
	Databaser: & Mnesia, Elastic Search, MySQL
\end{tabular}

\section{Övriga kunskaper}
\begin{tabular}{l p{3in}}
	Project metodik: & Scrum, Agile development
\end{tabular}

\section{Idell \\ verksamhet}
{\bf Träningsassistent,} TianLong Kungfu, Stockholm, Sverige \hfill 2005 -- 2006
\begin{itemize} \itemsep -2pt
\item Starta gruppträningar i Kungfu med uppvärmning och grundläggande övningar.
\item Ibland ledde jag hela sessioner av gruppträningar i Kungfu.
\item Hade hand om både barn, tonåringar och vuxna.
\end{itemize}

{\bf Köksvolontär,} Vässarölägret, Stockholm, Sverige \hfill Summer 2005
\begin{itemize} \itemsep -2pt
\item Jag var del av en av två grupper som förberedde maten för hela volontärsstaben på Vässarölägret.
\item Förberedde mat i storköket på Vässarölägret.
\end{itemize}

{\bf Sjöinstruktor,} Vässarölägret, Stockholm, Sverige \hfill Summer 2004
\begin{itemize} \itemsep -2pt
\item Instruera barn, tonåringar och vuxna i hur man seglar och paddlar kanot, inklusive genomgång av säkerhet.
\item Övervaka barn, tonåringar och vuxna när de hanterar Vässarölägrets segelbåtar och kanoter. Följa med scoutkårer/scoutlag på segelutflykter som lärare.
\end{itemize}

{\bf Styrelsemedlem,} Flemingsbergs scoutkår, Stockholm, Sverige \hfill 2004
\begin{itemize} \itemsep -2pt
\item Närvara vid styrelsemöten och agera för scoutkårens bästa.
\end{itemize}
{\bf Ledarassistent,} Flemingsbergs scoutkår, Stockholm, Sverige \hfill 2004
\begin{itemize} \itemsep -2pt
\item Assistera scoulagsledaren och hålla koll på barnen i laget.
\item Planera scoutmöten och utflykter för och med barnen.
\end{itemize}

\section{Referenser}
Tillhandahålls vid förfrågan.

\end{resume}
\end{document}
