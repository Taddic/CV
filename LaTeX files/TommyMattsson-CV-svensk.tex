\documentclass[purpleprocv]{procv}
\begin{document}
    \begin{procv-twocolumns}
        % --------------------------------------
        % First column, spans over all pages
        % --------------------------------------
        \profilepic{profilepicture}
        \aboutme{Profil}
        {Mjukvaruutvecklare med stor vilja att lära mig nya saker! Att 
        ställa frågor för att lära mig nya saker är självklart för mig. 
        Jag har gedigen erfarenhet av Erlang och Elixir och utökar för 
        närvarande mina kunskaper med Python och Golang. Under min karriär 
        har jag arbetat mycket med verktyg som Git, Jenkins, Docker och Bash, 
        för att nämna några. Jag är engagerad i att skriva högkvalitativ, 
        vältestad kod som är tydlig och lättillgänglig 
        för andra utvecklare.}
        \begin{skills}{Programmering}
            \skillrow{Erlang}{4.5}
            \skillrow{Elixir}{3}
            \skillrow{Python}{2.5}
            \skillrow{Docker}{2}
            \skillrow{Jenkins}{2}
            \skillrow{Html/CSS}{3.5}
            \skillrow{Javascript}{2.5}
            \skillrow{Bash}{3}
            \skillrow{\LaTeX}{3.5}
        \end{skills}
        \links{Länkar}{
            \website{taddic.com/}
            \linkedin{linkedin.com/in/tommymattsson}
            \github{github.com/Taddic}
        }
        \begin{skills}{Språk}
            \skillrow{Svenska}{5}
            \skillrow{Engelska}{4.5}
            \skillrow{Franska}{1}
        \end{skills}
        \references{Referenser}{Referenser lämnas gärna på begäran.}
        \begin{personal}{Personliga detaljer}
            \detail{Nationalitet}{Svensk}
            \detail{Födelsedag}{15 juni 1987}
        \end{personal}
        \hobbies{Hobbys}
        {På fritiden tycker jag om att hålla på med acroyoga och 
        cheerleading, spela tv-spel och brädspel, samt hålla på med
        hantverk som snickeri, sömnad med mera.}
        %
        % --------------------------------------
        % Second colum, spans over all pages
        % --------------------------------------
        \switchcolumn % This needs to be at the top of the second column
        %
        \header{Tommy}{Matsson}{Mjukvaruutvecklare}{
            \input{contactinfo.tex}
        }
        \experience{Arbetslivserfarenhet}{
          \exprow{Entryfy}{Stockholm, Sweden}{Sep 2025 - Ongoing}{Elixir, Phoenix LiveView}{System Developer}
          {Fokuserad på backendutveckling med Elixir och frontendutveckling med Phoenix LiveView."
          \\[3pt]
          Entryfy är en leverantör av övervakningslösningar som erbjuder kompletta larmsystem och passerkortshantering anpassade för företag.}
          \exprow{Erlang Solutions}{Stockholm, Sverige}{Jun 2023 - Feb 2025}{Erlang, Elixir, Github-actions}{Mjukvaruutvecklare}
          {Hjälpte kunder att designa och 
          leverera tillförlitliga system inom BEAM-ekosystemet. Bidrog med råd kring 
          projektarkitektur och best practices, samtidigt som jag skrev högkvalitativ 
          Erlang-kod anpassad efter kundernas affärsbehov. Under denna period lärde 
          jag mig även Elixir, vilket visar på anpassningsförmåga och mitt engagemang 
          för att utveckla min kompetens för att bättre kunna hjälpa kunder. Min tid 
          hos Erlang Solutions gav mig möjlighet att bidra till projekt där skalbarhet, 
          robusthet och prestanda var avgörande.
          \\[3pt]
          Erlang Solutions är ett konsultföretag välkänt för sin djupa expertis inom Erlang 
          och Elixir, och är engagerat i att främja och stödja BEAM-ekosystemet.}
          \exprow{Klarna}{Stockholm, Sverige}{Sep 2021 - Maj 2023}{Erlang, Docker, Jenkins, Ansible}{Mjukvaruutvecklare}
          {Medlem i ett Scrum-team "Kred Core" bestående av fyra personer, med ansvar för att underhålla det äldre monolitiska 
          systemet "Kred". Även om jag bidrog med 
          utveckling av Erlang-kod, låg huvuddelen av mitt arbete inom DevOps och kontinuerlig 
          integration. Det innefattade bland annat att underhålla Jenkins-instanser samt 
          att skriva automatiseringsskript med hjälp av Ansible och Docker.
          \\[3pt]
          Klarna erbjuder betallösningar för e-handel, där en av de mest populära tjänsterna 
          är "Handla nu, få produkterna, betala senare."}
          \exprow{Telia Company}{Uppsala, Sverige}{Feb 2017 - Aug 2021}{Erlang, Docker, Jenkins}{Mjukvaruutvecklare}
          {Huvudansvaret var utveckling av Erlang-kod inom ett Scrum-team  bestående av åtta personer. Jag ansvarade för 
          integrationen mot Facebook och tog proaktivt initiativ till att förbättra bygg- och 
          deploymentsprocesser, vilket drev teamets övergång mot DevOp och continuous 
          integration. Med hjälp av Docker satte jag upp nya Jenkins-instanser för att 
          kontinuerligt köra tester som inte hade körts på länge. Jag underhöll även dessa 
          Jenkins-miljöer och gav stöd till andra team som använde dessa miljöer.
          \\[3pt]
          Telia är ett telekommunikationsföretag som erbjuder internet, mobiltelefoni och 
          andra digitala tjänster. Jag arbetade inom den avdelning som utvecklade "ACE", 
          en kundtjänstplattform med funktioner såsom telefonköer, e-posthantering, 
          integration med sociala medier med mera.}
          \exprow{Cisco Systems}{Stockholm, Sverige}{Aug 2014 - Jan 2017}{}{Supporttekniker}
          {Inom Cisco hade jag samma roll som tidigare på Tail-f, som förvärvades av Cisco. 
          Under denna period fortsatte jag att använda min expertis som supportingenjör.}
          \exprow{Tail-f Systems}{Stockholm, Sverige}{Mar 2014 - Jul 2014}{YANG, XPath, Erlang, C, Python, Bash}{Supporttekniker}
          {Medlem i ett team som tillhandahöll teknisk support för att hjälpa kunder med företagets 
          mjukvaruprodukter, ConfD och NCS. Mina arbetsuppgifter inkluderade att hantera 
          supportärenden via ett online-system och att erbjuda teknisk support för 
          att lösa kundrapporterade problem.
          \\[3pt]
          Tail-f Systems utvecklade två mjukvaruprodukter, ConfD och NCS. ConfD installeras 
          på nätverksutrustning för att förenkla skapandet av användargränssnitt, medan NCS är 
          utformat för att avsevärt effektivisera nätverkshantering.}
          \exprow{Ericsson}{Shanghai, Kina}{Jan 2013 - Jul 2013}{Emacs, Erlang, Git}{Master thesis position}
          {Masteruppsats med fokus på innehållsfiltrering. Projektet innefattade implementering 
          av ICAP-protokollet (IETF RFC 3507) i Erlang, avsett för integration i ett 
          verktyg för lasttester utvecklat av Ericsson. Jag studerade även 
          telekommunikationsarkitekturer såsom GPRS, WCDMA, LTE, PGW och MME.
          \\[3pt]
          Ericsson är en ledande global leverantör av telekommunikationsutrustning och tjänster.}
          \exprow{Klarna}{Stockholm, Sverige}{Jun 2009 - Okt 2009\newline Jun 2008 - Aug 2008}{Erlang, SVN, Emacs, Bash}{Mjukvaruutvecklare}
          {Under mina universitetsstudier hade jag möjlighet att arbeta på Klarna under två 
          somrar. Jag fick chansen att lära mig och få erfarenhet i Erlang. Mina 
          arbetsuppgifter inkluderade att implementera mindre funktioner i Erlang för att 
          stödja pågående projekt. 
          \\[3pt]
          Klarna erbjuder betallösningar för e-handel, där en av de mest populära tjänsterna 
          är "Handla nu, få produkterna, betala senare."}
        }
        \experience{Utbildning}{
          \exprow{Uppsala University}{Uppsala, Sverige}{Jan 2014}{}{M.Sc, Computer Science}
          {Uppsala universitet samarbetade med Tongji-universitetet för en gemensam 
          masterutbildning med fokus på nätverk, distribuerade system och mjukvaruutveckling.}
          \exprow{Tongji University}{Shanghai, Kina}{Jun 2013}{}{M.Sc, Software Engineering}
          {Tongji-universitetet samarbetade med Uppsala universitet för en gemensam 
          masterutbildning med inriktning på mjukvaruutveckling och kinesiska språket.}
          \exprow{Uppsala University}{Uppsala, Sverige}{May 2012}{}{B.Sc, Computer Science}
          {Fokuserade på att utveckla programmeringsfärdigheter, designa algoritmer och 
          analysera deras effektivitet, samt andra grundläggande principer inom datavetenskap.}
        }
        \experience{Ideell verksamhet}{
          \exprow{Styrelsemedlem}{Jan 2014 - Jan 2019}{www.uppsalagf.se}{}{Uppsala Gymnastikförening}
          {Ansvarade för föreningens IT, deltog i styrelsemöten och utförde uppgifter enligt 
          beslut tagna under mötena.}
          \exprow{Tränare i gymnastik}{Jan 2014 - Jan 2019}{www.uppsalagf.se}{}{Uppsala Gymnastikförening}
          {Planerade och ledde gymnastikträningar för både barn och vuxna. Gav individuell 
          instruktioner för att stödja gymnasternas utveckling utifrån deras specifika behov och mål.}
        }
    \end{procv-twocolumns}
\end{document}