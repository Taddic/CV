% LaTeX file for resume
% This file uses the resume document class (res.cls)

\documentclass[margin]{res}
\usepackage[utf8]{inputenc}
% the margin option causes section titles to appear to the left of body text
\textwidth=5.2in % increase textwidth to get smaller right margin
%\usepackage{helvetica} % uses helvetica postscript font (download helvetica.sty)
%\usepackage{newcent}   % uses new century schoolbook postscript font

\begin{document}

\name{Curriculum Vitae - Tommy Mattsson\\[12pt]} % the \\[12pt] adds a blank line after name

\address{{\bf Contact information}\\
\input{contact.tex}
linkedin.com/in/tommymattsson\\
github.com/Taddic\\
}

\address{{\bf Address}\\
  \input{address.tex}\\
}

\begin{resume}

\section{Experience}
{\bf Support Engineer,} Cisco Systems, Stockholm, Sweden \hfill 2014-08 -- Ongoing
 \begin{itemize} \itemsep -2pt  % reduce space between items
 \item Cisco aquired Tail-f. The position itself is the same, see
   the post below for details.
 \end{itemize}

{\bf IT-Support,} Tail-f Systems, Stockholm, Sweden \hfill 2014-03 -- 2014-07
 \begin{itemize} \itemsep -2pt  % reduce space between items
 \item Provide technical support for the software ``ConfD'' and ``NCS''
   developed by Tail-f systems.
 \item Working with technologies such as YANG modeling, NETCONF,
   XML, XPath, Erlang, C, Java, Python
 \end{itemize}

{\bf Master thesis position,} Ericsson, Shanghai, China \hfill 2013-01 -- 2013-07
 \begin{itemize} \itemsep -2pt  % reduce space between items
 \item Title of the thesis was ``Content filtering''. I were to design
   and implement ICAP (IETF RFC 3507) in Erlang for a load test tool
   developed by Ericsson.
 \item Learned about telecommunication architecture, such as: \\
 GPRS, WCDMA, LTE, SGSN, GGSN, 3G, 2G, PGW, SGW, MME
 \end{itemize}

{\bf Teacher assistant,} Uppsala University, Uppsala, Sweden \hfill 2013-06 -- 2013-08\\
{\bf Teacher assistant,} Uppsala University, Uppsala, Sweden \hfill 2012-06 -- 2012-08
 \begin{itemize} \itemsep -2pt  % reduce space between items
 \item Teacher assistant for a class named ``Peer-To-Peer computing''.
 \item Improve previous Erlang labs with a better structure and user interface.
 \item Help students at lab sessions and over e-mail. Grade students lab exercises.
 \end{itemize}

{\bf IT-Support,} Uppsala University, Uppsala, Sweden \hfill 2010-10 -- 2012-08
 \begin{itemize} \itemsep -2pt  % reduce space between items
 \item Help students with IT related issues regarding UNIX and Windows
   computer halls at the IT department of Uppsala University.
 \item Prepare new hardware, install operating systems and software for UNIX/Linux, Mac and Windows.
 \end{itemize}

{\bf System developer,} Klarna AB, Stockholm, Sweden \hfill 2009-06 -- 2009-10\\
{\bf System developer,} Klarna AB, Stockholm, Sweden \hfill 2008-06 -- 2008-08
 \begin{itemize} \itemsep -2pt  % reduce space between items
 \item Developing features in Erlang for Klarnas' online invoice system.
 \end{itemize}

\section{Education}
{\bf M.Sc,  Computer Science,} Uppsala University, Uppsala, Sweden \hfill \textbf{Ongoing} \\
Cooperation with Tongji University, Shanghai, China. \\
Focusi on networking, distributed systems and software engineering.

{\bf M.Sc, Software Engineering,} Tongji University, Shanghai, China \hfill \textbf{Ongoing} \\
Cooperation with Uppsala University, Uppsala, Sweden.\\
Focus on Software engineering and Chinese language.

{\bf B.Sc, Computer Science,} Uppsala University, Uppsala, Sweden \hfill May 2012 \\
General computer science

\newpage
\section{Project}
{\bf Sensor search engine,} Uppsala University, Uppsala, Sverige \hfill 2013-09 -- 2014-01
 \begin{itemize} \itemsep -2pt  % reduce space between items
 \item Implementation was done mainly in Erlang and Ruby on Rails in
   a class called ``Project Computer Science'' for my master degree.
 \item Open source project hosted on github. \\
 https://github.com/projectcs13/sensor-cloud\\
 https://github.com/projectcs13/sensor-cloud-website
 \end{itemize}

\section{Language \\ Skills}
   \begin{tabular}{l p{3in}}
    Swedish: & Native proficiency \\
    English: & Professional working proficiency \\
    French:  & Elementary proficiency \\
    Chinese:  & Elementary proficiency
 \end{tabular}

% Tabulate Computer Skills; p{3in} defines paragraph 3 inches wide
\section{Programming language Skills}
   \begin{tabular}{l p{3in}}
    Functional Programming: & Erlang, Haskell, Lisp \\
    Imperative programming: & Java, Python, PHP, C, C++ \\
    Web programming: & JavaScript, HTML, CSS, Ruby on Rails \\
    UNIX/Linux: & Bash scripting, Makefile \\
    App development: & Android API \\
    Other: & LaTeX (this resume is created using LaTeX)
 \end{tabular}

\section{Computer software}
\begin{tabular}{l p{3in}}
  IDE: & Eclipse, Netbeans \\
  CVS: & Git, SVN \\
  Webservers: & Apache, YAWS, Mochiweb \\
  Editors: & Emacs \\
  Databases: & Mnesia, Elastic Search, MySQL
\end{tabular}

\section{Other skills}
\begin{tabular}{l p{3in}}
  Project methodology: & Scrum, Agile development
\end{tabular}

\section{Non-profit activities}
{\bf Training assistant,} TianLong Kungfu, Stockholm, Sweden \hfill 2005 -- 2006
\begin{itemize} \itemsep -2pt
\item Start off Kungfu group training sessions with warm up and basic exercises. Lead entire kungfu sessions on occasion.
\item Occasionally lead entire Kungfu sessions for adolescents, teens and adults.
\end{itemize}

{\bf Kitchen Volounteer,} Vässarö scouting camp, Stockholm, Sweden \hfill Summer 2005
\begin{itemize} \itemsep -2pt
\item I was part of one of the two teams who prepared food for all of the volounteers at the Vässarö scouting complex.
\item Prepared food in the large-scale kitchen of the Vässarö complex.
\end{itemize}

{\bf Sailboat instructor,} Vässarö scouting camp, Stockholm, Sweden \hfill Summer 2004
\begin{itemize} \itemsep -2pt
\item Instruct and supervise adolescents, teens and adults in the usage and safety of sailboats and canoes.
\item Accompanied scout troops on sailing excursions as teacher and supervisor.
\end{itemize}

{\bf Board member,} Flemingsbergs scout troop, Stockholm, Sweden \hfill 2004
\begin{itemize} \itemsep -2pt
\item Attend board meetings and look out for the overall interests of the scout troop.
\end{itemize}
{\bf Leader assistant,} Flemingsbergs scout troop, Stockholm, Sweden \hfill 2004
\begin{itemize} \itemsep -2pt
\item Assist the troop leader and look after the adolescents.
\item Plan scout meetings and excursions for the adolescents.
\end{itemize}

\section{References}
Provided on demand.

\end{resume}
\end{document}
